
In \autoref{sec:roots} and \autoref{sec:cyclotomic}, we learned about the definition and properties of the $\mu$-th roots of unity and the $\mu$-th cyclotomic polynomial over complex numbers (i.e., $X \in \mathbb{C}$) as follows: 

\begin{itemize}
\item \textbf{The $\bm \mu$-th roots of unity} are the solutions for $X^\mu = 1$ over $X \in \mathbb{C}$ (complex numbers). The formula for the $\mu$-th root of unity is $X = e^{2 \pi i k / \mu}$ for all integer $k$ such that $0 \leq k \leq \mu - 1$. 
\item \textbf{The primitive $\bm \mu$-th roots of unity (denoted as $\bm \omega$)} are those $\mu$-th roots of unity whose order (\autoref{subsec:order-def}) is $\mu$ (i.e., $\omega^{\mu} = 1$ and $\omega^{\frac{\mu}{2}} \neq 1$). 
\item Given any primitive $\mu$-th roots of unity $\omega$, it can generate all primitive $\mu$-th roots of unity by computing $\omega^{k'}$ such that $k'$ is an integer $0 < k' < \mu$ and $\textsf{gcd}(k', \mu) = 1$ (Theorem~\ref*{subsec:roots-theorem}.4 in \autoref{subsec:order-theorem}). 
\item \textbf{The $\bm \mu$-th cyclotomic polynomial} is defined as a polynomial whose roots are the primitive $\mu$-th roots of unity. That is, \[ \Phi_{\mu}(x) = \prod_{\omega \in P({\mu})} (x - \omega) = \prod_{\substack{0 \leq k \leq {\mu}-1,\\ \text{gcd}(k, {\mu}) = 1}} (x - \omega^k) \]
\end{itemize}

In this section, we will explain the $\mu$-th cyclotomic polynomial over $X \in \mathbb{Z}_p$ (ring), which is structured as follows: 

\begin{tcolorbox}[title={\textbf{\tboxdef{\ref*{sec:cyclotomic-polynomial-integer-ring}} Roots of Unity and Cyclotomic Polynomial over Rings $\mathbb{Z}_p$}}]


\begin{itemize}
\item \textbf{The $\bm \mu$-th roots of unity (denoted as $\bm \omega$)} are the solutions for $X^\mu \equiv 1 \bmod p$. Note that these solutions are not $X = \omega^{2 \pi i k / \mu}$ (the formula for the solutions over $X \in \mathbb{C}$). 
\item \textbf{The primitive $\bm \mu$-th roots of unity} are defined as those $\mu$-th roots of unity whose order (\autoref{subsec:order-def}) is $\mu$ (i.e., $\omega^{\mu} \equiv 1 \bmod p$, and $\omega^{\lceil \frac{\mu}{2} \rfloor} \not\equiv 1 \bmod p$). 
\item Given any primitive $\mu$-th roots of unity $\omega$, it can generate all primitive $\mu$-th roots of unity by computing $\omega^{k'}$ such that $k'$ is an integer $0 < k' < \mu$ and $\textsf{gcd}(k', \mu) = 1$.
\item \textbf{The $\bm \mu$-th cyclotomic polynomial} is defined as a polynomial whose roots are the primitive $\mu$-th roots of unity. That is, \[ \Phi_{\mu}(x) = \prod_{\omega \in P({\mu})} (x - \omega) = \prod_{\substack{0 \leq k \leq {\mu}-1,\\ \text{gcd}(k, {\mu}) = 1}} (x - \omega^k) \]
\end{itemize}

\end{tcolorbox}

\begin{table}[h] %usepackage{array} 
\begin{tabular}{|c||c||c|}
\hline \hline
& \textbf{Polynomial over $\bm{X} \bm{\in} \bm{\mathbb{C}}$} & \textbf{Polynomial over $\bm{X} \in \bm{\mathbb{Z}}_{\bm{p}}$} \\ 
& \textbf{(Complex Number)} & \textbf{(Ring)} \\ \hline \hline
\textbf{Definition}&All $X \in \mathbb{C}$ such that $X^\mu = 1$, (which are&All $X \in \mathbb{Z}_p$ such that $X^\mu \equiv 1 \bmod p$\\
\textbf{of the}&computed as $X = e^{2 \pi i k / \mu}$ for integer $k$&\\
\textbf{$\bm \mu$-th}&where $0 \leq k \leq \mu - 1$)&\\
\textbf{Root of Unity}&&\\\hline
\textbf{Definition}&Those $\mu$-th roots of unity $\omega$ such that&Those $\mu$-th roots of unity $\omega$ such that\\
\textbf{of the}&$\omega^{\mu} = 1$, and $\omega^{\frac{\mu}{2}} \neq 1$&$\omega^{\mu} \equiv 1 \bmod p$, and $\omega^{\frac{\mu}{2}} \not\equiv 1 \bmod p$\\
\textbf{Primitive}&&\\
\textbf{$\bm \mu$-th}&&\\
\textbf{Root of}&&\\
\textbf{Unity}&&\\\hline
\textbf{Definition}&\multicolumn{2}{|c|}{The polynomial whose roots are the $\mu$-th primitive roots of unity as follows:}\\
\textbf{of the}&\multicolumn{2}{|c|}{$ \Phi_{\mu}(x) = \prod_{\omega \in P(\mu)} (x - \omega) $  \text{ } (see Definition~\ref*{subsec:cyclotomic-def} in \autoref{subsec:cyclotomic-def})}\\
\textbf{$\bm \mu$-th}&\multicolumn{2}{|c|}{}\\
\textbf{Cyclotomic}&\multicolumn{2}{|c|}{}\\
\textbf{Polynomial}&\multicolumn{2}{|c|}{}\\\hline
\textbf{Finding}&For $\omega = e^{2 \pi i/ \mu}$, compute all ${\omega}^k$ such that&Find one satisfactory $\omega$ that is a root of\\
\textbf{Primitive}&$0 < k < \mu $ and $\textsf{gcd}(k, \mu) = 1$&the $\mu$-th cyclotomic polynomial, and\\
\textbf{$\bm \mu$-th}&(Theorem~\ref*{subsec:roots-theorem}.4 in \autoref{subsec:roots-theorem})&compute all $\omega^k \bmod p$ such that\\
\textbf{Roots of}&&$0 < k < \mu $ and $\textsf{gcd}(k, \mu) = 1$\\
\textbf{Unity}&&\\\hline\hline
\end{tabular}
\caption{The roots of unity and cyclotomic polynomials over $X \in \mathbb{C}$ v.s. over $X \in \mathbb{Z}_p$}
\label{tab:cyclotomic-polynomial-comparison}
\end{table}

Note that in the $\mu$-th cyclotomic polynomial in both cases of over $X \in \mathbb{C}$ and over $X \in \mathbb{Z}_p$, each of their roots $\omega$ (i.e., the primitive $\mu$-th root of unity) has the order $\mu$ (i.e., $\omega^{\mu} = 1$ over $X \in \mathbb{C}$, and $\omega^{\mu} \equiv 1 \bmod p$ over $X \in \mathbb{Z}_p$). Also note that each root $\omega$ can generate all roots of the $\mu$-th cyclotomic polynomial by computing $\omega^{k'}$ such that $\textsf{gcd}(k', \mu) = 1$.

\autoref{tab:cyclotomic-polynomial-comparison} compares the properties of the roots of unity and the $\mu$-th cyclotomic polynomial over $X \in \mathbb{C}$ (complex numbers) and over $X \in \mathbb{Z}_p$ (ring).






