%\textbf{- First Read:} 
%\href{https://e.math.cornell.edu/people/belk/numbertheory/CyclotomicPolynomials.pdf}{Fields and Cyclotomic Polynomials}

\subsection{Definitions}
\label{subsec:group-def}


\begin{tcolorbox}[title={\textbf{\tboxdef{\ref*{subsec:group-def}} Group}}]
\noindent \textbf{\underline{Set Elements}}
\begin{itemize}
\item \textbf{Set ($\mathbb{S}$):} A bundle of elements: $\mathbb{S} = \{a, b, c, \ldots\}$
\item \textbf{Set Operations $\bm{(+, \cdot)}$:} We consider two operations on $\mathbb{S}$ between any two elements $a, b \in \mathbb{S}$ as operands: addition $(+)$ and multiplication $(\cdot)$
\item \textbf{Additive Identity ($0_{(+)}$ often written $0$):} An element $i \in \mathbb{S}$ is an additive identity if for all $a \in \mathbb{S}$, $i + a = a$.
\item \textbf{Multiplicative Identity ($1_{(\cdot)}$ often written $1$):} An element $i \in \mathbb{S}$ is a multiplicative identity if for all $a \in \mathbb{S}$, $i \cdot a = a$
\item \textbf{Additive Inverse ($a^{-1}_{(+)}$):} For each $a \in \mathbb{S}$, its additive inverse $a^{-1}_{(+)}$, often written $-a$, is defined as an element such that $a + a^{-1}_{(+)} = 0_{(+)}$ (i.e., additive identity)
\item \textbf{Multiplicative Inverse ($a^{-1}_{(\cdot)}$):} For each $a \in \mathbb{S}$ that is invertible with respect to $(\cdot)$, its multiplicative inverse $a^{-1}_{(\cdot)}$, often written $a^{-1}$, is defined as an element such that $a \cdot a^{-1}_{(\cdot)} = 1_{(\cdot)}$ (i.e., multiplicative identity)
\end{itemize}

$ $

\noindent \textbf{\underline{Element Operation Features}}
\begin{itemize}
\item \textbf{Closed:} A set $\mathbb{S}$ is closed under the $(+)$ operation if for every $a, b \in \mathbb{S}$, it is the case that $a + b \in \mathbb{S}$. Likewise, a set $\mathbb{S}$ is closed under the $(\cdot)$ operation if for every $a, b \in \mathbb{S}$, it is the case that $a \cdot b \in \mathbb{S}$. 
\item \textbf{Associative:} $(a + b) + c = a + (b + c)$
\item \textbf{Commutative:} $a + b = b + a$
\item \textbf{Distributive:} When both $(+)$ and $(\cdot)$ are defined, $a \cdot (b + c) = (a \cdot b) + (a \cdot c)$
\end{itemize}


$ $

\noindent \textbf{\underline{Group Types}}
\begin{itemize}
\item \textbf{Semigroup:} A semigroup is a set of elements which is closed and associative on a single operation ($+$ or $\cdot$)
\item \textbf{Monoid:} A monoid is a semigroup, plus it has an identity element $e$, which returns the other operand over the set operation.

(e.g., $0$ is the identity element for $+$ operator, $1$ is the identity element for the $\cdot$ operator)
\item \textbf{Group:} A group is a monoid, and every element has an inverse with respect to the operation.
\item \textbf{Abelian Group:} An abelian group is a group, plus its operation is commutative.
\end{itemize}
\end{tcolorbox}

\subsection{Examples}
\label{subsec:group-ex}

$\mathbb{Z}$ (i.e., the set of all integers) is an abelian group under addition ($+$), because:
\begin{itemize}
\item \textbf{Closed:} For any integer $a, b \in \mathbb{Z}$, $a + b = c$ is also an integer ($\in \mathbb{Z}$).
\item \textbf{Associative:} For any integer $a, b, c \in \mathbb{Z}$, $(a + b) + c = a + (b + c)$.
\item \textbf{Identity:} The additive identity is 0, because for any $a \in \mathbb{Z}$, $a + 0 = a$.
\item \textbf{Inverse:} For each $a \in \mathbb{Z}$, its additive inverse is $-a$, as $a + (-a) = 0$.
\item \textbf{Commutative: } For any integer $a, b \in \mathbb{Z}$, $a + b = b + a$.
\end{itemize}

$ $

\noindent $\mathbb{Z}$ is a monoid under multiplication ($\cdot$), because:
\begin{itemize}
\item \textbf{Closed:} For any integer $a, b \in \mathbb{Z}$, $a \cdot b = c$ is also an integer ($\in \mathbb{Z}$).
\item \textbf{Associative:} For any integer $a, b, c \in \mathbb{Z}$, $(a \cdot b) \cdot c = a \cdot (b \cdot c)$.
\item \textbf{Identity:} The multiplicative identity is 1, because for any $a \in \mathbb{Z}$, $a \cdot 1 = a$.
\item \textbf{NO Inverse:} For an integer $a \in \mathbb{Z}$, its multiplicative inverse is $\dfrac{1}{a}$, but this is not necessarily an integer ($\notin \mathbb{Z}$), therefore not every element has a multiplicative inverse, so $(\mathbb{Z},\cdot)$ is not a group (though it is a monoid).
\end{itemize}
$ $

\noindent $\mathbb{R}^\times$ (i.e., the set of all nonzero real numbers) is an abelian group under multiplication ($\cdot$), because:
\begin{itemize}
\item \textbf{Closed:} For any real number $a, b \in \mathbb{R}^\times$, $a \cdot b = c$ is also a real number ($\in \mathbb{R}^\times$).
\item \textbf{Associative:} For any real number $a, b, c \in \mathbb{R}^\times$, $(a \cdot b) \cdot c = a \cdot (b \cdot c)$.
\item \textbf{Identity:} The multiplicative identity is 1, as for any real number $a \in \mathbb{R}^\times$, $a \cdot 1 = a$.
\item \textbf{Inverse:} For each real number $a \in \mathbb{R}^\times$, its multiplicative inverse is $\dfrac{1}{a}$, which is a real number ($\in \mathbb{R}$).
\end{itemize}